% !TEX root = ../thesis.tex
\documentclass[
  ukcolor,       % remove for less red,
  BCOR=12mm,     % 12mm binding corrections, adjust to fit your binding
  parskip=half,  % new paragraphs start with half line vertical space
  open=any,      % chapters start on both odd and even pages
  cleardoublepage=plain,  % no header/footer on blank pages
]{technical/agethesis}


% This project is meant to be run with LuaLaTeX, but it should also work with pdfLaTeX
% When setting commands that are only available in LuaLaTeX, use the following:
%   \ifx\luatexversion\undefined
        % pdfLaTeX
%   \else
        % LuaLaTeX
%   \fi


% Warning, if another latex run is needed
\usepackage[aux]{rerunfilecheck}

% just list chapters and sections in the toc, not subsections or smaller: 1, otherwise 2 or 3
\setcounter{tocdepth}{2}

% acronyms
\usepackage{acronym}

%------------------------------------------------------------------------------
%------------------------------ Fonts, Unicode, Language ----------------------
%------------------------------------------------------------------------------

%\ifx\luatexversion\undefined
    % pdfLaTeX
%    \usepackage[utf8]{inputenc}
%    \usepackage[T1]{fontenc}
%\else
    % luaLaTeX
    \usepackage{fontspec}
    \defaultfontfeatures{Ligatures=TeX}  % -- becomes en-dash etc.
%\fi


% load english (for abstract) and ngerman language
% the main language has to come last
\usepackage[ngerman, american]{babel}

% intelligent quotation marks, language and nesting sensitive
\usepackage[autostyle]{csquotes}

% microtypographical features, makes the text look nicer on the small scale
\usepackage{microtype}

%------------------------------------------------------------------------------
%------------------------ Math Packages and settings --------------------------
%------------------------------------------------------------------------------

\usepackage{amsmath}
\usepackage{amssymb}
\usepackage{mathtools}

 %Enable Unicode-Math and follow the ISO-Standards for typesetting math. Diable for pdfLaTeX.
%\ifx\luatexversion\undefined
    % pdfLaTeX -- do nothing
%\else
    \usepackage[
        math-style=ISO,
        bold-style=ISO,
        sans-style=italic,
        nabla=upright,
        partial=upright,
        warnings-off={mathtools-colon,mathtools-overbracket}, % suppress some unnecessary warnings
        ]{unicode-math}
 %   \setmathfont{Latin Modern Math} % or any other OpenType math font
%\fi

% nice, small fracs for the text with \sfrac{}{}
\usepackage{xfrac}


%------------------------------------------------------------------------------
%---------------------------- Numbers and Units -------------------------------
%------------------------------------------------------------------------------

\usepackage[
  locale=US,
  exponent-product=\cdot,
  separate-uncertainty=true,
  per-mode=symbol-or-fraction,
]{siunitx}

\DeclareSIUnit\angstrom{\text {Å}}
\DeclareSIUnit\pixel{px}  
\DeclareSIUnit\frames{frames}

\usepackage{physics}
\usepackage{chemformula} % chemical formulas, enables \ch{}

%\ifx\luatexversion\undefined
    % pdfLaTeX
%\else
    \renewcommand{\vb}[1]{\symbf{#1}}
%\fi

%------------------------------------------------------------------------------
%-------------------------------- tables  -------------------------------------
%------------------------------------------------------------------------------

\usepackage{booktabs}       % \toprule, \midrule, \bottomrule, etc

%------------------------------------------------------------------------------
%-------------------------------- graphics -------------------------------------
%------------------------------------------------------------------------------

\usepackage{graphicx}
% currently broken
% \usepackage{grffile}

% allow figures to be placed in the running text by default:
\usepackage{scrhack}
\usepackage{float}
\floatplacement{figure}{htbp}
\floatplacement{table}{htbp}

% keep figures and tables in the section
\usepackage[section, below]{placeins}

% allows to include PDFs as full pages
\usepackage{pdfpages}

% Set the PDF Version of this document to 1.7 (1.4 is the current default)
% This is needed so that PDFs with Version >1.5 can be included
% Disable for pdfLaTeX
%\ifx\luatexversion\undefined
    % pdfLaTeX -- do nothing
%\else
    \pdfvariable minorversion=7
%\fi

\usepackage{cprotect} % to include verb in caption \cprotect\caption{... \verb|trackpy|...}

%------------------------------------------------------------------------------
%--------------------------- TikZ and Pgfplots --------------------------------
%------------------------------------------------------------------------------

%\usepackage{tikz}
%\usepackage{pgfplots}
%\usetikzlibrary{positioning, math, intersections, calc}

%\usetikzlibrary{external}
%\tikzexternalize[prefix=TikZ/]
%\pgfplotsset{compat=1.18}


%------------------------------------------------------------------------------
%---------------------- customize list environments ---------------------------
%------------------------------------------------------------------------------

\usepackage{enumitem}

%------------------------------------------------------------------------------
%------------------------------ Bibliographie ---------------------------------
%------------------------------------------------------------------------------

\usepackage[
  backend=biber,   % use modern biber backend
  autolang=hyphen, % load hyphenation rules for if language of bibentry is not
                   % german, has to be loaded with \setotherlanguages
                   % in the references.bib use langid={en} for english sources
	%sorting=nyt,  % try out those 4 lines
	%style=alphabetic,
%%	style=numeric-comp,
	%backend=biber
]{biblatex}
\addbibresource{technical/Particle_Transport.bib}  % the bib file to use
\DefineBibliographyStrings{german}{andothers = {{et\,al\adddot}}}  % replace u.a. with et al.


% Last packages, do not change order or insert new packages after these ones
\usepackage[
    pdfusetitle,
    unicode,
    % linkbordercolor=white,
    % citebordercolor=white, % if color is needed: ukruby
]{hyperref}
\hypersetup{colorlinks=false, linkbordercolor={white}}

%\usepackage[
	%colorlinks=true,
	%linkcolor=black,
	%citecolor=black,
	%urlcolor=black,
%]{hyperref}

\usepackage{bookmark}
\usepackage[shortcuts]{extdash}

%------------------------------------------------------------------------------
%-------------------------    Angaben zur Arbeit   ----------------------------
%------------------------------------------------------------------------------

\author{Yahya Shubbak}
\title{Traveling wave magnetophoresis for Lab-on-a-Chip applications}
\date{\today} %TODO
\birthplace{Kassel}
\supervisor{Unter~Anleitung~von~Prof.~Arno~Ehresmann~und~Dr.~Rico~Huhnstock}
\chair{AG Ehresmann}
\division{Institut für Physik\\FB 10 - Mathematik und Naturwissenschaften}
\thesisclass{Doktor der Naturwissenschaften (Dr. rer. nat.)} % Change to Master of Science for master thesis
\submissiondate{\today} %TODO
\firstcorrector{Prof.~Dr.~Arno Ehresmann} %TODO
\secondcorrector{Prof.~Dr.~Zweitgutachter} %TODO

% uk logo on top of the titlepage
\titlehead{\centering\includegraphics[width=6cm]{Abbildungen/logos/uk-logo.pdf}}

\graphicspath{
{Abbildungen/},
{Abbildungen/01_Einfuehrung/,
{Abbildungen/02_Theorie/},
{Abbildungen/03_Exp/},
{Abbildungen/04_Ergebnisse/},
{Abbildungen/05_Auswertung/},
{Abbildungen/06_Diskussion/}
}}
     % if you want to compile with pdfLaTeX, make sure you visit the three locations marked by pdfLaTeX and comment out the corresponding lines

\begin{document}
\frontmatter

\maketitle

% Gutachterseite
\makecorrectorpage

% hier beginnt der Vorspann, nummeriert in römischen Zahlen
\input{content/00_abstract.tex} % Optional
\tableofcontents
 


\mainmatter

Ich heiße dich herzlich willkommen, YaSh!
Nun müsste alles funktionieren.
Damit du mit \texttt{latexmk} arbeitest, musst du lediglich in die Konsole \texttt{latexmk} eingeben, um zu kompilieren.
Damit du kompilierst, und die PDF sehen kannst, musst du nichts weiteres tun, als \texttt{latexmk -pv} einzugeben.
Um kontinuierlich zu kompilieren, kannst du \texttt{latexmk -pvc} verwenden.
Du wirst feststellen, dass \texttt{latexmk} sehr viel schneller kompiliert als overleaf. 
Und wenn du dann noch einen vernünftigen Text-Editor willst, komm zum neovim-Kult. Du holst die Zeit, die du für's Lernen braucht in
ca. 7 Jahren wieder ein. Und nebenbei haben wir auch Kekse! Copilot schlägt mir vor: \enquote{und Bier.} 
Astaghfirullah! 


% Hier beginnt der Inhalt mit Seite 1 in arabischen Ziffern
\include{content/01_introduction}
% !TEX root = ../thesis.tex
\chapter{Theoretical background} \label{sec:theorie}
\section{Magnetism}\label{sec:theory_magnetism}
\subsection{Atomic magnetism}\label{sec:theory_atomic_magnetism}
\subsection{Magnetic interactions}\label{sec:theory_magnetic_interactions}
\subsection{Thin film magnetism}\label{sec:theory_thin_film_magnetism}

\section{Magnetic anisotropy}\label{sec:theory_magnetic_anisotropy}

\section{Exchange bias}\label{sec:theory_exchange_bias}

\section{DLVO-Forces}\label{sec:theory_dlvo}
\subsection{Electrostatic intercations}
\subsection{Van der Waals}

\section{Traveling wave magnetophoresis}
In this section, the theoretical background of traveling wave magnetophoresis and its implications are explained.
We will start with ...

\subsection{Particle transport mechanism}


\subsection{Close to surface transport}

% !TEX root = ../thesis.tex
\chapter{Experimental methods}\label{sec:experimental}
\section{Particle transport setup}
The setup consists of the components depicted in Figure x.
First, the sample stage is placed in the middle point of a 3D Helmholtz coil setup cosisting of pairs in $x$, $y$, and $z$ directions. 
Due to the geometry of the setup, the coils have different windings. 
The $x$ coils have n windings, the $y$ couls have m windings and the $z$ couls have l windings. producing x, y, z mT at 1 A, respectively.
The coils are individually driven by a power supply and adressed by a lab view software controlled by an NI box.
This allows for trapiziodal fields to be applied as used for the experiments.
The coil setup is then placed under a Zeiss Axiotop reflective microscope.
This is important as the samples are opaic and a traditional transmissive microsocpe won't work.
The microscope was fitted with a 20 x and 50x objektive depending on the experiment conduted. 
Finally, to record the experiments, the highspeed camera EoSense 2.0 MCX12-CM was mounted on top. 
This monochromatic camera is cabable of recording 1920x1080 pixles resolution videos at around 2000 frames. 
The data aquisitoin is done via a frame grabber installed in the aquisiton PC.



\section{Transport substrate preparation}
To create the transport substrates, a 15mmx15mm naturally oxidized Si substrate is cleaned with isopropanole and dried in a stream of N2 before placing in a RF sputtering mashine. 
There, Ar plasma is ignited while a magnetic field is applied.
...
Films of x,y,z nm are sputtered onto the Si substrate creating the magnetic layer stack.
This happened in the presense of a magnetic field of x mT, giving the initial direction of the magnetic moments of the ferromagnet.

The substrate is then placed in a small furnace that heats the sample above the Neel temperature and while keeping below the curie temperature, freely rotating the AF.
As there is a magnetic field in the direction of the magneic moment of the FM, the topmost layer of atoms of the AF allign to the direction of the FM, creating a pinning called exchange coupling.

The substrate is then covered with a lithographic mask which, in this case, masekd 5µm thin stripes.
After developing  in xy, the substrate is placed in a homemade He Ion blaster. 
This introduces both energy and defect in the unmasked portion of the substrate, allowing the magnetic moments to rotate.
While this is happening, a magnetic field opposing the magentic moment of the substrate then turns the now freed magnetic moments. 
After He irradiation, the magentic moments are now pinned by the Af in the oppoisng directoin, creating the stripe domain pattern we know and love. 
\subsection{Thin film sputter deposition}
Sputter deposition is most commonly used to achieve monolayer accuracy in thickness, creating the magnetic thin film stack \cite{Seshan2012}.    
Sputtering, in a nutshell, is the process of accelerating ionized atoms, often argon, into a surface of desired material called a target, which can be metals or insulators, with high enough kinetic energy to physically remove some target atoms and condense them onto a substrate.
This creates thin films with minimum thicknisses of x nm



\subsection{Ion bombardment induced magnetic patterning}

\section{Magnetic characterization}

\subsection{MOKE magnetometry}

\subsection{Vibrating sample magnetomitry}

\subsection{Magnetic force microscopy}

\subsection{Magnetophoretic velocity measurement away from any wall}

\subsection{Dynamic light scattering: Zeta potential}




\chapter{Particle transport evaluation software}
\section{Trackpy and the Crocker-Grier algorithm}
The following chapter will shed light on the most important computational steps of analysis using the fantastic Python library \verb|trackpy| version 0.6.2\cite{Allan2024}.



In general, the evaluation consists of location of features and subsequent temporal linking.
The process is as follows:
\paragraph{Feature location}
The image series is loaded, containing the particles as bright features against a dark background.
Features will be called from hereafter particles for simplicity. 
Initial parameters are tested until particles are found using \verb|trackpy.locate|.
The fundamental parameters are \verb|diameter|,\verb| minmass|, and \verb|separation|. 
The \verb|diameter| is given in pixels and will look for particles of spherical dimensions in a given box of $d^2$ pixels.
For particles of unequal dimensions, a tuple can be given as \verb|(z, y, x)| or \verb|(y,x)|. 
The \verb|minmass| parameter represents the minimum integrated brightness of bright (dark) particles against a dark (bright) background. 
It is therefore important to specify whether the images were taken using dark-field or bright-field microscopy.
Since dark-field will be the default for tracking, \verb|invert| as a \verb|trackpy.locate| parameter would be needed for bright field images.
Here, \verb|trackpy| uses the Crocker-Grier algorithm to locate the particles, where first a bandpass filter is applied to the image to remove both long wavelength background variations such as uneven lighting and short wavelength noise such as camera noise\cite{Crocker1996}.
Then, local maxima are identified as particle centers using the brightness of the particles, and the brightness and size of each particle is measured\cite{Crocker1996}.
A good starting point for feature finding for images of bright, well-resolved spherical particles of several micrometer in diameter against a dark, static background, using a 20$\times$ objective and a 1920$\times$1080 resolution camera, is:  
\begin{verbatim} 
    diameter    :   13
    minmass     : 2000
    separation  :   12
\end{verbatim}
An example of such an image is found in Fig. \ref{fig:darkfield_tracekd_COOH}.

\begin{figure}[bth]
    \centering
    \includegraphics[width=.8\linewidth]{Abbildungen/Theory/COOH_DestH2O_PMMA700nm_2mT_100ms_1000fps_0_features_2800_with_insert.png}
    \cprotect\caption{Example dark-field microscope image of bright particles marked as features using \verb|trackpy|.
    The parameters used were: diameter 13, minmass 2000, and separation 12.}
    \label{fig:darkfield_tracekd_COOH}
\end{figure}

\verb|trackpy.locate| allows far more options to be tweaked to enhance the linking after the particles have been found.
These include, but are not limited to:
% \verb|threshold|, which clips the bandpass result below the given value, default is 1. 
\verb|separation|, which is a minimum separation in pixels between particles with the default value of \verb|diameter +1|. 
\verb|percentile|, which is a threshold value of the particles' brightness compared to the mean overall brightness of all pixels in the image, that has to be exceeded for them to be recognized as such. 

\paragraph{Linking}
\verb|trackpy| then starts linking the particles together to form coherent trajectories. 
For that, a second set of parameters can be optimized, including
\verb|search range|, which is the maximum distance particles can move between frames, and \verb|memory|, which is the maximum number of frames during which a particle can vanish, then reappear nearby, and be considered the same particle. 
% \verb|adaptive_stop|:
% \verb|adaptive_step|:
Again, \verb|trackpy| uses the Crocker-Grier algorithm, which uses proximity among particles from one frame to the next, based on Brownian motion of the particle to diffuse without interacting with other particles\cite{Crocker1996}.
This Brownian motion Ansatz then works with the \verb|search range| to allow for particles to move a certain distance between frames in a directed manner as they carry a velocity and are not only diffusing randomly.  

The result is a table in H5 format containing \verb|frame|, \verb|x|, and \verb|y| position per given particle.
From that, a map of all trajectories can be plotted, which can be seen in Fig. \ref{fig:Traj_no_filter}.

\paragraph{Mean particle velocity - or trajecto maps}
To evaluate the velocity, two conversion factors are implemented, as the temporal information is given in frames rather than seconds, and the spatial information in pixels, rather than (micro)meters.  
Conversion to seconds is self-explanatory, as the chosen frame rate of 1000 FPS translates to a temporal resolution of 1 millisecond per frame. 
To convert from pixels to micrometers, the pixel pitch resulting from the combination of the camera sensor and microscope objective must be measured.
Since a monochrome sensor is used, no color filter for the pixels is present, meaning that the calculation is a simple translation from the distance between two given pixels into length.
For the used camera and the two used microscope objectives, the following conversion factors are used: \SI{0.5}{\micro\meter\per\pixel} for the $20\times$objective, and \SI{0.2}{\micro\meter\per\pixel} for the $50\times$objective. 
\begin{figure}
    \centering
    \includegraphics[width=0.75\linewidth]{Abbildungen/Theory/Traj_no_filter.pdf}
    \caption{Trajectory map of COOH MPs at $\nu=\SI{3.01}{\Hz}$, no filtering applied.}
    \label{fig:Traj_no_filter}
\end{figure}

Then, a series of filtering and outlier removal must be conducted due to the polydisperse nature of the particles and the linking algorithm of \verb|trackpy|. 
Fig. \ref{fig:Traj_no_filter} shows a trajectory map of \ch{COOH} MPs at $\nu=\SI{3.01}{\Hz}$, prior to filtering.
It contains 2567 unique particles. 
It becomes obvious that some trajectories are fragmented, as highlighted in the red box. 
This happens when MPs are too close together as they move along $x$, which causes the \verb|diameter| of the two MPs to overlap in the linking step.
The algorithm then looses track of one or both MPs, causing fragmented trajectories.
Very short, individual trajectories are produced out of mostly two MPs together.
These otherwise valid trajectories have to be removed, as they lack spatial and temporal information and cause the number of trajectories to be unphysically high. 
Fig. \ref{fig:Traj_2000filter} shows a trajectory map after such a filtering step.
Here, trajectories that are shorter than \SI{2000}{\frames}, which represents \SI{40}{\percent} of the total duration, are removed.
Removing shorter trajectories already eliminates the very short ..
Filtering out less heavy, e.g., trajectories $<\SI{500}{\frames}$, already removes the fragmented trajectories.
However, it was found that the theoretical fit of the overall mean velocity is represented better when filtering heavier. 
This could be avoided if a smaller density of MPs have been choosen for the experiment and will be discussed later.
\begin{figure}
    \centering
    \includegraphics[width=0.75\linewidth]{Abbildungen/Theory/Traj_2000filter.pdf}
    \caption{Trajectory map of COOH MPs at $\nu=\SI{3.01}{\Hz}$, MPs present for less than 2000 frames removed.}
    \label{fig:Traj_2000filter}
\end{figure}


Alongside the fragmentation, which is an caused by the linking algorith of \verb|trackpy|, Fig. \ref{fig:Traj_2000filter} shows a physical artifact as well.
Even after filtering for short temporal trajectories, short spatial trajectories remain.
These are essential in the non-linear regime where MPs can not follow the moving potential energy landscape linearly, some are only oscillating above a DW. 
In the linear regime however (Eq. TODO), MPs do not perform any oscillatory motion around a DW and do follow the changing potential energy landscape.
Trajectories should therefore have the length in accordance with Eq. TODO with the given duration of \SI{5}{\s} and a period length $d=\SI{10}{\micro\m}$ be \SI{150.6}{\micro\m}.
This, of course, is only true for MPs who are present for the whole duration, e.g., in a initial left to right transport (positive $x-$direction), MPs start near $x=0$ and move to higher $x$ before the phase relation change flips the transport direction and the MPs are tranpsorted back.
Shorter trajectories are possible if MPs leave the ROI by starting at higher $x$, or enter it at a given time.
However, short trajectories that present themselves as circles, are, at least in the linear regime, actually pinned and therefore, falsly lower the mean velocity of the ensemble in the linear regime.
These are MPs often stuck to defects on the substrate in some form and are, in fact, able to move, which spares them from being identified as static background.  
Therefore, another filtering step is performed.
In Fig. \ref{fig:Traj_2000filter_20micometer}, these oscillating MPs are removed by filtering for MPs that have been transported for less than \SI{20}{\micro\m} which, in this case is twice the period $d$. 
Filtering for MPs $<\SI{10}{\micro\m}$ tend to keep some of the osccilating MPs, if the total oscillation duration lasts for more than \SI{10}{\micro\m}, which is the reason \SI{20}{\micro\m} is choosen.
\begin{figure}
    \centering
    \includegraphics[width=0.75\linewidth]{Abbildungen/Theory/Traj_2000filter_20micometer.pdf}
    \caption{Trajectory map of COOH MPs at $\nu=\SI{3.01}{\Hz}$, MPs present for less than 2000 frames and transported for less than \SI{20}{\micro\meter} removed.}
    \label{fig:Traj_2000filter_20micometer}
\end{figure}

Table \ref{Table:filtering_summary} show a small summary of the filtering steps and the resulting number of trajectories kept. 

\begin{table}[htb]
\caption{Summary of filtering steps and resulting number of trajectories and mean velocity for \ch{COOH} MPs at \SI{5}{\Hz} \SI{30}{\Hz} driving frequency, representing linear and non-linear regimes respectively.}
\label{Table:filtering_summary}
\begin{tabular}{lllll}
Experiment & Regime    & Trajectories removed                    & n Trajectories & v mean (expected) / $\si{\micro\m\per\s}$ \\
\SI{5}{\Hz}       & linear     & No filtering                            & xxx            & y                        \\
           &            & $<\SI{2000}{\frames}$                    & xxx            & y                        \\
           &            & $<\SI{20}{\micro\m}$                        & xxx            & y                        \\
           &            & $<\SI{2000}{\frames}$ + $<\SI{20}{\micro\m}$ & xxx            & y                        \\
\SI{30}{\Hz}      & non-linear & No filtering                            & xxx            & y                        \\
           &            & $<\SI{2000}{\frames}$                  & xxx            & y                       
\end{tabular}
\end{table}
% !TEX root = ../thesis.tex
% \chapter{Results and discussion}
\chapter{Surface-surface interaction based separation of magnetic particles}
\section{Characterization of superparamagnetic particles}
\subsection{Zeta potential}
\subsection{Magnetophoretic velocity}
\section{Transport dynamics close to the surface}
\subsection{Mean ensemble velocity}
\subsection{Steady-state velocity?}

\chapter{Transport of magnetic particles bound to proteins?}
\section{Sample preparation?}
\subsection{Physiological conditions for proteins?}
\subsection{Influence on Zeta potential?}
\subsection{Transport dynamics?}

\chapter{Particle transport on tissue}
\section{Sample preparation}
\section{Transport dynamics}

\chapter{White light interferometer for absolute height determination?}





% !TEX root = ../thesis.tex
\chapter{Diskussion}

% !TEX root = ../thesis.tex
\chapter{Summary and outlook}


\appendix
% Hier beginnt der Anhang, nummeriert in lateinischen Buchstaben
\include{content/a_anhang}

\backmatter
\printbibliography

\cleardoublepage
% From ttps://www.uni-kassel.de/fb10/organisation/pruefungsbuero/download-formulare
\section{Eigenständigkeitserklärung}
Hiermit bestätige ich, dass ich die vorliegende Arbeit selbständig verfasst und keine
anderen als die angegebenen Hilfsmittel benutzt habe. Die Stellen der Arbeit, die dem
Wortlaut oder dem Sinn nach anderen Werken (dazu zählen auch Internetquellen)
entnommen sind, wurden unter Angabe der Quelle kenntlich gemacht.

\begin{minipage}{0.4\textwidth}
    \includegraphics[width=0.4\textwidth]{unterschrift.jpg}\\
    Abdulrahman Shubbak
\end{minipage}\hfill
\begin{minipage}{0.4\textwidth}
    \vspace*{20ex}
    Kassel, den \today
\end{minipage}



% !TEX root = ../thesis.tex
\chapter{Eigenständigkeitserklärung}
Hiermit bestätige ich, dass ich die vorliegende Arbeit selbständig verfasst und keine
anderen als die angegebenen Hilfsmittel benutzt habe. Die Stellen der Arbeit, die dem
Wortlaut oder dem Sinn nach anderen Werken (dazu zählen auch Internetquellen)
entnommen sind, wurden unter Angabe der Quelle kenntlich gemacht.

\begin{minipage}{0.4\textwidth}
    \vspace*{10ex}
    Yahya Shubbak
\end{minipage}\hfill
\begin{minipage}{0.4\textwidth}
    \vspace*{10ex}
    Kassel, den \today
\end{minipage}



\end{document}
