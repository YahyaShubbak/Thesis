\chapter{Particle transport evaluation software}
\section{Trackpy and its implication}
The following chapter will shed light on the most important steps of analysis using the fantastic Python library \verb|trackpy| version 0.6.2\cite{Allan2024}.



The evaluation pipeline is as follows:

The image series is loaded, containing the particles as bright features against a dark background.
Features will be called from hereafter particles for simplicity. 
Initial parameters are tested until particles are found using \verb|trackpy.locate|.
The fundamental parameters are \verb|diameter|,\verb| minmass|, and \verb|separation|. 
The \verb|diameter| is given in pixels and will look for particles of spherical dimensions in a given box of $d^2$ pixels.
For particles of unequal dimensions, a tuple can be given as \verb|(z, y, x)| or \verb|(y,x)|. 
The \verb|minmass| parameter represents the minimum integrated brightness of bright (dark) particles against a dark (bright) background. 
It is therefore important to specify whether the images were taken using dark-field or bright-field microscopy.
Since dark-field will be the default for tracking, \verb|invert| as a \verb|trackpy.locate| parameter would be needed for bright field images.
A good starting point for feature finding for images of bright, well-resolved spherical particles against a dark, static background is:  
\begin{verbatim} 
    diameter    :   13
    minmass     : 2000
    separation  :   12
\end{verbatim}
An example of such an image is found in Fig. \ref{fig:darkfield_tracekd_COOH}.

\begin{figure}[bth]
    \centering
    \includegraphics[width=.8\linewidth]{Abbildungen/Theory/COOH_DestH2O_PMMA700nm_2mT_100ms_1000fps_0_features_2800_with_insert.png}
    \cprotect\caption{Example dark-field microscope image of bright particles marked as features using \verb|trackpy|.
    The parameters used were: diameter 13, minmass 2000, and separation 12.}
    \label{fig:darkfield_tracekd_COOH}
\end{figure}

\verb|trackpy.locate| allows far more options to be tweaked to enhance the linking after the particles have been found.
These include, but are not limited to:
% \verb|threshold|, which clips the bandpass result below the given value, default is 1. 
\verb|separation|, which is a minimum separation in pixels between particles with the default value of \verb|diameter +1|. 
\verb|percentile|, which is a threshold value of the particles' brightness compared to the mean overall brightness of all pixels in the image, that has to be exceeded for them to be recognized as such. 


\verb|trackpy| then starts linking the particles together to form coherent trajectories. 
For that, a second set of parameters can be optimized, including
\verb|search range|, which is the maximum distance particles can move between frames, and \verb|memory|, which is the maximum number of frames during which a particle can vanish, then reappear nearby, and be considered the same particle. 
% \verb|adaptive_stop|:
% \verb|adaptive_step|:

The result is a table in H5 format containing \verb|frame|, \verb|x|, and \verb|y| position per given particle.
From that, a map of all trajectories can be plotted, which can be seen in Fig. \ref{fig:Traj_no_filter}.


To evaluate the velocity, two conversion factors are implemented, as the temporal information is given in frames rather than seconds, and the spatial information in pixels, rather than (micro)meters.  
Conversion to seconds is self-explanatory, as the chosen frame rate of 1000 FPS translates to a temporal resolution of 1 millisecond per frame. 
To convert from pixels to micrometers, the pixel pitch resulting from the combination of the camera sensor and microscope objective must be measured.
Since a monochrome sensor is used, no color filter for the pixels is present, meaning that the calculation is a simple translation from the distance between two given pixels into length.
For the used camera and the two used microscope objectives, the following conversion factors are used: \SI{0.5}{\micro\meter\per\pixel} for the $20\times$objective, and \SI{0.2}{\micro\meter\per\pixel} for the $50\times$objective. 


Then, a series of filtering and outlier removal must be conducted due to the polydisperse nature of the particles and the linking algorithm of \verb|trackpy|. 
Fig. \ref{fig:Traj_no_filter} shows a trajectory map of \ch{COOH} MPs at $\nu=\SI{3.01}{\Hz}$, prior to filtering.
It contains 2567 unique particles. 
It becomes obvious that some trajectories are fragmented, as highlighted in the red box. 
This happens when MPs are too close together as they move along $x$, which causes the 
Since this is in the linear regime (Eq. TODO), MPs do not perform any oscillatory motion around a DW.
Trajectories should therefore have the length in accordance with Eq. TODO with the given duration of \SI{5}{\s} and a period length $d=\SI{10}{\micro\m}$ be \SI{150.6}{\micro\m}.
This, of course, is only true for MPs who are present for the whole duration.
Shorter trajectories are possible if MPs leave the ROI or enter it at a given time.
However, short trajectories in the center of the ROI, which often present themselves as circles, are unphysical.
These are MPs often stuck to defects on the substrate in some form and are, in fact, able to move, which spares them from being identified as static background.  
\begin{figure}
    \centering
    \includegraphics[width=0.75\linewidth]{Abbildungen/Theory/Traj_no_filter.pdf}
    \caption{Trajectory map of COOH MPs at $\nu=\SI{3.01}{\Hz}$, no filtering applied.}
    \label{fig:Traj_no_filter}
\end{figure}

\begin{figure}
    \centering
    \includegraphics[width=0.75\linewidth]{Abbildungen/Theory/Traj_2000filter.pdf}
    \caption{Trajectory map of COOH MPs at $\nu=\SI{3.01}{\Hz}$, MPs present for less than 2000 frames removed.}
    \label{fig:Traj_2000filter}
\end{figure}

\begin{figure}
    \centering
    \includegraphics[width=0.75\linewidth]{Abbildungen/Theory/Traj_2000filter_20micometer.pdf}
    \caption{Trajectory map of COOH MPs at $\nu=\SI{3.01}{\Hz}$, MPs present for less than 2000 frames and transported for less than \SI{20}{\micro\meter} removed.}
    \label{fig:Traj_2000filter_20micometer}
\end{figure}
