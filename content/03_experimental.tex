% !TEX root = ../thesis.tex
\chapter{Experimental methods}\label{sec:experimental}
\section{Particle transport setup}
The setup consists of the components depicted in Figure x.
First, the sample stage is placed in the middle point of a 3D Helmholtz coil setup cosisting of pairs in $x$, $y$, and $z$ directions. 
Due to the geometry of the setup, the coils have different windings. 
The $x$ coils have n windings, the $y$ couls have m windings and the $z$ couls have l windings. producing x, y, z mT at 1 A, respectively.
The coils are individually driven by a power supply and adressed by a lab view software controlled by an NI box.
This allows for trapiziodal fields to be applied as used for the experiments.
The coil setup is then placed under a Zeiss Axiotop reflective microscope.
This is important as the samples are opaic and a traditional transmissive microsocpe won't work.
The microscope was fitted with a 20 x and 50x objektive depending on the experiment conduted. 
Finally, to record the experiments, the highspeed camera EoSense 2.0 MCX12-CM was mounted on top. 
This monochromatic camera is cabable of recording 1920x1080 pixles resolution videos at around 2000 frames. 
The data aquisitoin is done via a frame grabber installed in the aquisiton PC.



\section{Transport substrate preparation}
To create the transport substrates, a 15mmx15mm naturally oxidized Si substrate is cleaned with isopropanole and dried in a stream of N2 before placing in a RF sputtering mashine. 
There, Ar plasma is ignited while a magnetic field is applied.
...
Films of x,y,z nm are sputtered onto the Si substrate creating the magnetic layer stack.
This happened in the presense of a magnetic field of x mT, giving the initial direction of the magnetic moments of the ferromagnet.

The substrate is then placed in a small furnace that heats the sample above the Neel temperature and while keeping below the curie temperature, freely rotating the AF.
As there is a magnetic field in the direction of the magneic moment of the FM, the topmost layer of atoms of the AF allign to the direction of the FM, creating a pinning called exchange coupling.

The substrate is then covered with a lithographic mask which, in this case, masekd 5µm thin stripes.
After developing  in xy, the substrate is placed in a homemade He Ion blaster. 
This introduces both energy and defect in the unmasked portion of the substrate, allowing the magnetic moments to rotate.
While this is happening, a magnetic field opposing the magentic moment of the substrate then turns the now freed magnetic moments. 
After He irradiation, the magentic moments are now pinned by the Af in the oppoisng directoin, creating the stripe domain pattern we know and love. 
\subsection{Thin film sputter deposition}
To achieve high-quality thin films with monolayer accuracy, sputter deposition is most commonly used\cite{Seshan2012}.    
Sputtering, in a nutshell, is the process of accelerating ionized atoms, often \ac{Ar}, into a surface of desired material referred to as the target, which can be metals or insulators, with high enough kinetic energy to physically remove some target atoms and condense them onto a substrate.
\ch{Ar} is usually used as it is cost-effective, chemically inert, and its atomic mass is in the same order as desired metals that are used in thin film applications\cite{Seshan2012}.  
Since the bombardment of ions physically removes target atoms, this technique is classified as a \ac{PVD} technique\cite{Seshan2012}. 
However, compared to evaporation, sputter deposition produces a high-energy flux with high surface mobility, allowing atoms to condense into smooth, dense, conformal, and continuous films more easily than evaporated films\cite{Seshan2012}.
When non-inert gases, such as oxygen or nitrogen, are used, the process is then called reactive sputtering, which is typically used to deposit oxides or nitrides\cite{Seshan2012}.

Linked to the apparatus used, three main sputtering mechanisms are available: \ac{DC} diode sputtering, \ac{RF} sputtering, and magnetron sputtering.
They all share the same core principles, and the aufbau is similar, which can be seen in Figure x. 
For \ac{DC} diode sputtering, two plates working as anode and cathode connected to a power supply are placed in a vacuum.
The anode is composed of the substrate to be coated, and the cathode is the material to be sputtered\cite{Frey2015}.
\ac{Ar}, as the typical sputtering gas, is introduced in the pressure range of \SI{e-3}{mbar}, and a voltage across the plates is applied\cite{Seshan2012}.  
Above a certain breakdown voltage, a plasma is ignited, creating free electrons and accelerating \ch{Ar+} towards the negative cathode in the electric field.
Since the electrons are much lighter than the ions, they experience a much higher velocity, which results in a thin sheath layer next to the cathode, an area of electron depletion causing most of the potential drop between anode and cathode\cite{Seshan2012}. 
There, the acceleration of the ions is greatest, right befor they impact into the target\cite{Paetzold2004}. 
Secondary electrons are created upon impact, which are accelerated through the sheath layer towards the anode, ionizing more \ac{Ar} in the path, which sustains the plasma.   
Electrons that do not collide with \ch{Ar} are absorbed by the anode and the apparatus walls, not contributing to ionization\cite{Seshan2012}.
Generally, the sputter yield is proportional to the power supplied to create the plasma.
Ionization cross section peaks for electron energies of about \SI{100}{eV} and then drops, meaning that the process cannot be scaled by simply applying more power\cite{Seshan2012}. 
Additionally, in \ac{DC} diode sputtering, only electrically conductive materials can be used as target materials, as the buildup of positive charge of the \ch{Ar+}-ions could not be dissipated, which would lead to the gradual loss of the accelerating potential, terminating the sputtering altogether\cite{Paetzold2004}. 
For this reason, \ac{RF} sputtering was utilized in this work.

In \ac{RF} sputtering, the \ac{DC} power supply is replaced by an \ac{AC} \ac{RF} source\cite{Seshan2012}.
The supplied \ac{RF} is typically \SI{13.56}{MHz} or some multiple\cite{Seshan2012}.
This \ac{RF} power couples to the electron motion in the plasma, resulting in longer residence times in the plasma, higher collisional ionization, and higher plasma densities\cite{Seshan2012}.
To maximize the energy transfer, impedance-matching hardware is installed to tune the output impedance of the power supply to impedance-match the plasma impedance\cite{Seshan2012}.
Additionally, the \ac{RF} sputtering allows for sputtering insulators such as silicon dioxide.
This is due to the \ac{AC} applied to the diodes, preventing charge buildup\cite{Seshan2012}. 

\subsection{Ion bombardment induced magnetic patterning}

\section{Magnetic characterization}

\subsection{MOKE magnetometry}

\subsection{Vibrating sample magnetomitry}

\subsection{Magnetic force microscopy}

\subsection{Magnetophoretic velocity measurement away from any wall}

\subsection{Dynamic light scattering: Zeta potential}



