% !TEX root = ../thesis.tex
\chapter{Experimental methods}\label{sec:experimental}
\section{Particle transport setup}
The setup consists of the components depicted in Figure x.
First, the sample stage is placed in the middle point of a 3D Helmholtz coil setup cosisting of pairs in $x$, $y$, and $z$ directions. 
Due to the geometry of the setup, the coils have different windings. 
The $x$ coils have n windings, the $y$ couls have m windings and the $z$ couls have l windings. producing x, y, z mT at 1 A, respectively.
The coils are individually driven by a power supply and adressed by a lab view software controlled by an NI box.
This allows for trapiziodal fields to be applied as used for the experiments.
The coil setup is then placed under a Zeiss Axiotop reflective microscope.
This is important as the samples are opaic and a traditional transmissive microsocpe won't work.
The microscope was fitted with a 20 x and 50x objektive depending on the experiment conduted. 
Finally, to record the experiments, the highspeed camera EoSense 2.0 MCX12-CM was mounted on top. 
This monochromatic camera is cabable of recording 1920x1080 pixles resolution videos at around 2000 frames. 
The data aquisitoin is done via a frame grabber installed in the aquisiton PC.



\section{Transport substrate preparation}
\subsection{Thin film sputter deposition}
To create the transport substrates, a 15mmx15mm naturally oxidized Si substrate is cleaned with isopropanole and dried in a stream of N2 before placing in a RF sputtering mashine. 
There, Ar plasma is ignited while a magnetic field is applied.
...
Films of x,y,z nm are sputtered onto the Si substrate creating the magnetic layer stack.
This happened in the presense of a magnetic field of x mT, giving the initial direction of the magnetic moments of the ferromagnet.

The substrate is then placed in a small furnace that heats the sample above the Neel temperature and while keeping below the curie temperature, freely rotating the AF.
As there is a magnetic field in the direction of the magneic moment of the FM, the topmost layer of atoms of the AF allign to the direction of the FM, creating a pinning called exchange coupling.

The substrate is then covered with a lithographic mask which, in this case, masekd 5µm thin stripes.
After developing  in xy, the substrate is placed in a homemade He Ion blaster. 
This introduces both energy and defect in the unmasked portion of the substrate, allowing the magnetic moments to rotate.
While this is happening, a magnetic field opposing the magentic moment of the substrate then turns the now freed magnetic moments. 
After He irradiation, the magentic moments are now pinned by the Af in the oppoisng directoin, creating the stripe domain pattern we know and love. 


\subsection{Ion bombardment induced magnetic patterning}

\section{Magnetic characterization}

\subsection{MOKE magnetometry}

\subsection{Vibrating sample magnetomitry}

\subsection{Magnetic force microscopy}

\subsection{Magnetophoretic velocity measurement away from any wall}



